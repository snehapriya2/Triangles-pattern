\documentclass[14pt]{beamer}
\title{TRIANGLE'S}
\date{\today}
\author{Bvrith \\ Team 55 \\ J Akshaya Varshini: 20WH1A0521: CSE \\ Shaik Ayesha: 20WH1A05H0: CSE \\ B Pavithra: 20WH1A6621: CSE AIML \\ G Krishna Prathibha: 20WH1A12B0: IT \\ G Sneha Priya: 20WH1A0481: ECE \\ K Nithya: 21WH5A6604: CSE AIML}
\usefonttheme{serif}
\usepackage{bookman}
\usepackage{hyperref}
\usepackage[T1]{fontenc}
\usepackage{graphicx}
\usecolortheme{orchid}
\beamertemplateballitem

\begin{document}
    \begin{frame}
        \titlepage
    \end{frame}
    \begin{frame}
	\frametitle{Introduction}
        \begin{itemize}
	    \item Trace the number of Triangle's that are formed by grid edges in the input picture.		
	\end{itemize}
    \end{frame}\centerline{\includegraphics{C:/Users/DRL/Pictures/Screenshots/Screenshot (674).png}}
    \begin{frame}
	\frametitle{Approach}
	\begin{itemize}
	    \item After examining the input pattern,whenever the symbols like 'x','-','/','\' are encountered in grid pattern ,the number of triangles are calculated by using the concepts of loops and grids.
			   
	    \item we take the rows and columns as input and we  give the grid pattern in any form.
	    \item The output gives us the number of triangles in  the given grid  
	\end{itemize}
    \end{frame}
    \begin{frame}
        \frametitle{Learnings}
	\begin{itemize}
	    \item Day1: \\ 1. We learnt about Latex \\ 2. we have created our Git repo in gitlab.com \\
	    \item Day2: \\ 1. We tried understand the problem statement by breaking it into segments and analyse it. \\ also tried to figure out other requirements of the program such as time complexity.
	    \\ 3. There are two Possibilities of solving the problem statement. \\ Using numpy,By applying Dynamic problem programming.
		    \end{itemize}
	    \end{frame}
	\begin{frame}
	\begin{itemize}
	       \item DAY3: \\ 1. Tried to analyse the input statements.\\ 2. Went through the basic concepts of python. \\ 3. Understood the concept of dynamic programming. \\ 4. Worked on the development of code. \\ 5. There is an concept called TURTLE GRAPHICS in python and checked if we could use it to give in the input pattern(i.e. the grid edge pattern).
	        
	
\end{itemize}	  
\end{frame} 
\begin{frame}
\begin{itemize}
      \item DAY4: \\ 1. We had a session on pushing and pulling code in GITLAB(from terminal or command prompt). \\ 2. Tried to develope the basic logic of the code. 
\end{itemize}
\end{frame} 
\begin{frame}
\begin{itemize}
      \item DAY5:\\1.All the team members worked on the code.\\2.We tried to analyze the input and produce the output \\3.Errors need to be corrected and We will try to conclude the code by today.
      \end{itemize}
      \end{frame}
    \begin{frame}
	\frametitle{Challenges}
        \begin{itemize}
	    \item Challanges we faced were(Day 1 and 2): \\1.installing and understanding latex \\2. understanding the problem statement and exploring multiple ways of solving.
	    \item Day 3: \\ 1. Facing problem in uploading the picture in latex. \\ 2. Understanding dynamic programming was a time taking process.
        \end{itemize}
   \end{frame}
   \begin{frame}
   \begin{itemize}
        \item Day 4: \\ 1. We encountered errors while running the code and developing the logic. \\ 2. We are struck at a point in the code and are unable to proceed. 
   
   \end{itemize}
   \end{frame}
    \begin{frame}
	\frametitle{GIT Repo}
	\begin{itemize}
     \item {\includegraphics[width =8cm, height = 7cm]{C:/Users/DRL/Pictures/Screenshots/Screenshot (676).png}}
	\end{itemize}
    \end{frame}
    \begin{frame}
	\frametitle{Statistics}
        \begin{itemize}
	     \item Number of Lines of Code : 19
	     \item Number of Functions : 1
        \end{itemize}
    \end{frame}
    \begin{frame}
	\frametitle{Demo/ Screen Shots}
	\begin{itemize}
	  {  \item Demo of the project}
	   { \includegraphics[width = 8cm ,height = 7cm]{C:/}}
	\end{itemize}
    \end{frame}
    \begin{frame}
	\begin{center}
	     END (or) THANK YOU
	\end{center}
    \end{frame}
\end{document}
