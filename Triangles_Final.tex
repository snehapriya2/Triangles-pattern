\documentclass{beamer}
\title{BVRIT HYDERABAD College Of Engineering For Women}
\date{\today}
\author{TRIANGLES\\ TEAM 55 }
\usefonttheme{serif}
\usepackage{bookman}
\usepackage{hyperref}
\usepackage[T1]{fontenc}
\usepackage{graphicx}
\usetheme{paloalto}
\usecolortheme{whale}
\setbeamercolor{titlelike}{parent=structure,bg=cyan}
\beamertemplateballitem

\begin{document}
\begin{frame}
    \titlepage
\end{frame}
    \begin{frame}
    \frametitle{Team members}
        \begin{itemize}
      \item J Akshaya Varshini: 20WH1A0521: CSE 
      \item Shaik Ayesha: 20WH1A05H0: CSE 
      \item B Pavithra: 20WH1A6621: CSE AIML
      \item G Krishna Prathibha: 20WH1A12B0: IT 
      \item G Sneha Priya: 20WH1A0481: ECE 
      \item K Nithya: 21WH5A6604: CSE AIML   
        \end{itemize}
   
    \end{frame}
    \begin{frame}
	\frametitle{Introduction}
        \begin{itemize}
	    \item Trace the number of Triangle's that are formed by grid edges in the input picture.	
	    \includegraphics{C:/Users/DRL/Pictures/Screenshots/Screenshot (674).png}	
	\end{itemize}
    \end{frame}
    \begin{frame}
        \frametitle{Sample output}
        \begin{figure}[t]
           {\includegraphics[width = 8cm , height = 4cm]{c:/Users/DRL/Pictures/Screenshots/Screenshot (696).png}}
           \end{figure}
     \end{frame}
     \begin{frame}
       
        
        \centerline{\includegraphics[width = 8cm , height = 4cm]{C:/Users/DRL/Pictures/Screenshots/Screenshot (697).png}}
        
    \end{frame}
    \begin{frame}
	\frametitle{Approach}
	\begin{itemize}
	    \item After examining the input pattern,whenever the symbols like 'x','-','/','\' are encountered in grid pattern ,the number of triangles are calculated by using the concepts of loops and grids.
			   
	    \item we take the rows and columns as input
	    \item The output gives us the number of triangles in  the given grid . 
	\end{itemize}
    \end{frame}
    \begin{frame}
        \frametitle{Learnings}
	\begin{itemize}
	    \item LaTeX.
	    \item Git Repo.
	    \item Concepts of dynamic Programming/Data structures.
	 \end{itemize}
    \end{frame}
    \begin{frame}
	\frametitle{Challenges}
        \begin{itemize}
	    \item Initially faced trouble while installing and Using Latex and took time to understand its Interface.
	    \item Understanding the conept of Data structures( Dynamic Programming).
	   
        \end{itemize}
   \end{frame}
    \begin{frame}
	\frametitle{GIT Repo}
	
     \centerline{\includegraphics[width =8cm, height = 7cm]{C:/Users/DRL/Pictures/Screenshots/Screenshot (676).png}}

    \end{frame}
    \begin{frame}
	\frametitle{Statistics}
        \begin{itemize}
	     \item Number of Lines of Code:101
	     \item Number of Functions:1
        \end{itemize}
    \end{frame}
    \begin{frame}
	\frametitle{Screen Shots of Code} 
	     \centerline{\includegraphics[width = 8cm ,height = 8cm]{C:/Users/DRL/Pictures/Screenshots/Screenshot (688).png}}
	
    \end{frame}
    \begin{frame}
      \centerline {\includegraphics[width = 8cm , height = 8cm]{C:/Users/DRL/Pictures/Screenshots/Screenshot (689).png}}
    \end{frame}
    \begin{frame}
      \centerline{\includegraphics[width = 8cm , height = 8cm]{C:/Users/DRL/Pictures/Screenshots/Screenshot (690).png}}
    \end{frame}
    \begin{frame}
       \centerline{\includegraphics[width = 8cm , height = 8cm]{C:/Users/DRL/Pictures/Screenshots/Screenshot (691).png}}
    \end{frame}
    \begin{frame}
	\frametitle{Output}
	   \centerline{\includegraphics[width = 8cm , height = 8cm]{C:/Users/DRL/Pictures/Screenshots/Screenshot (693).png}}
    \end{frame}
    \begin{frame}
    \frametitle{Reference}
    \begin{itemize}
    \item {https://youtu.be/0LqHpmzRHzk}
    \item {Geeks for Geeks}
    \item{python.org}
    \item {https://www.csc.kth.se/~austrin/icpc/finals2018solutions.pdf}
    \end{itemize}
    \end{frame}
\end{document}
